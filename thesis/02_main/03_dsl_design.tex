\chapter{Design of the DSL}
\label{cha:dsl-design}

As mentionned by Van Deursen and Klint
\cite{little_languages_little_maintenance}, one point is to cluster the
knowledge we acquire into a set of semantic notions and opertions on them. This
chapter will present the whole design of the APDL DSL. Firstly, we analyse all
the concept we need to have for a complete DSL design for our case-study. Those
concept include the data source, the transmission, the storage and finally the
visualisation of the data.

Then, we will present the design of the library which is implementing the
clustered information.

\section{Data source}
\label{sec:dsl-design-data-source}

The concept of data source include some different information :

\begin{itemize}
\item The source device
\item The type of data
\item The sampling
\end{itemize}

\subsection{Source device}
\label{subsec:dsl-design-source-device}

Firstly, we have to know that there is a lot of different existing device for
the IoT programming, some of them are just connected ones like bluethoot tracker
or smart device like fridge or door. For the purpose of the project we are just
going the take device that are sensor oriented. All we want is to gather
information for the environmment and communicate it further.

In order to manage a varous kind of device, we will use the PlatformIO tool.

\subsubsection{PlatformIO}
\label{sec:dsl-design-source-device-platformio}

TODO

\subsubsection{Source properties}
\label{sec:dsl-design-source-properties}

The major property of Source is his developement framework. A lot of various framework exist for embeded programming :
begin

\subsubsection{Type of data}
\label{sec:dsl-design-type-of-data}

\subsubsection{Sampling}
\label{sec:dsl-design-sampling}


TODO : write about arduino and raspeberry

%%% Local Variables:
%%% mode: latex
%%% TeX-master: "../thesis"
%%% End:

\newChapter{Domain specific language}
\label{cha:a-dsl}

A Domain specific language (DSL) is a programming language which the goal is
to provide a way to solve specific domain problem. For example, Matlab is a
domain specific language because, in the first place, it was for matrix
manipulation and mathematic.

What we call ``domain'' is a set of problem related together in the same scope.
For Matlab, those domain could be a matrix multiplication or the resolution of a
equantion system. Those to operation are related to numerical mathematic and
that is the ``domain'' of Matlab.

This chapter will introduce and discuss the question ``Why use a DSL ?'' by
exposing some advantages that \gls{APDL} could bring for non-IoT-developers.
Then we present the diffenrence between an internal and an external DSL and the
advantages and disadvantages for both of them. Then we will conclude this
chapter by exposing the development phase of a DSL describe by Van Deursen and
Klint \cite{little_languages_little_maintenance}.

\section{Why use a Dsl ?}
\label{sec:why-use-dsl}

TODO

\section{Internal and External Domain specific language}
\label{sec:internal_and_external_dsl}

TODO

\section{Advantages and Disadvantages}
\label{sec:dsl-advantages-disadvantages}

TODO

\section{Implementation of a Domain specific language}
\label{sec:implementation_of_a_dsl}

TODO

\section{Development phase of a Domain specific language}

TODO

%%% Local Variables:
%%% mode: latex
%%% TeX-master: "../thesis"
%%% End:

\newChapter{Domain specific language}
\label{cha:a-dsl}

\section{Introduction}
\label{sec:dsl_intro}

A \gls{DSL} is a programming language whose the goal is
to provide a way to solve specific domain problem. For example, Matlab is a
domain specific language created to simplify matrix calculus.

What we call ``domain'' is a set of problem related together in the same scope.
For Matlab, those domain could be a matrix multiplication or the resolution of a
equantion system. Those to operation are related to numerical mathematic and
that is the ``domain'' of Matlab.

This chapter will introduce and discuss the question ``Why use a DSL ?'' by
exposing some advantages that \gls{APDL} could bring for non-IoT-developers.
Then we present the diffenrence between an internal and an external DSL and the
advantages and disadvantages for both of them. After introducing some technics

\section{Why use a Dsl ?}
\label{sec:why-use-dsl}

TODO

\section{Internal and External Domain specific language}
\label{sec:internal_and_external_dsl}

TODO

\section{Implementation of a Domain specific language with Scala}
\label{sec:implementation_of_a_dsl}

TODO

\subsection{Internal Domain specific language}
\label{subsec:scala_internal_dsl}

TODO

\subsection{External Domain specific language}
\label{subsec:scala_external_dsl}

TODO

\subsection{The LMS approach}
\label{subsec:lms_approach}

TODO

\subsection{Parser combinator}
\label{subsec:parser_combinator}

TODO

%%% Local Variables:
%%% mode: latex
%%% TeX-master: "../thesis"
%%% End:

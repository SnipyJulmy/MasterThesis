\newChapter{Internet of Things Development}
\label{cha:iot_challenge}

\section{Introduction}
\label{sec:iot_challenge_intro}

\gls{IoT} has become a huge area of research and development over the past few
years. With the emergence of data science and the library for big data analysis
like \gls{Hadoop} and \gls{Spark}, the scientist want's more and more data to
analyse.

Recovering environment data with embedded hardware is present from the start of
the 60's\cite{Community2017}. In contrary to standard computer systems, embedded
systems and \gls{IoT} systems suffers of several challenges.

This chapter introduce the challenges of the \gls{IoT} development, from the device
itself to the software implementation. Then, we will expose the development of
\gls{IoT} project using general programming languages and \gls{DSL}.

\section{Internet of Things challenges}
\label{sec:iot_challenges}

\subsection{Heterogeneity}
\label{sec:heterogeneity}

As the first challenge for the system development in \gls{IoT}, we should
mention the disparity of the devices. The heterogeneity of the objects does not
allow them to interact with each other \cite{midgar}. To connect them we need a
common interface like a central server or a common description language.

There is a lot of various device and development framework in the world of
embedded hardware : Arduino, Raspberry PI, Mbed, SPL, Simba, Teensy, and so
on. Each of them with specfic properties and specific programming language.

\subsection{Optimisation}
\label{sec:iot_challenge_optimisation}

One big limitation for \gls{IoT} system development is the power management. It
directly affects the algorithms we can use and means that the solutions we
create would always be the power-optimized one \cite{Sneps-Sneppe2016}. In
general, the power management can be done by the operating, which is a luxury
to have in most part of the available devices.

We have to mention in this context a technics named \gls{DPM}. The idea is to
shut down the devices when they are doing nothing and turn them on when they
need to receive / send data. In fact, it's kind of a replacement for the
\gls{OS}. But using such a technics could cause latency and additinal development.

\subsection{Connectivity}
\label{sec:connectivity}

In order to communicate each other, and as previusly announce in section
\ref{sec:heterogeneity}, \gls{IoT} devices need to communicates together trough
a standard interface. Developing such interface require experience and knowledge
in software development and programming language as well as a wide knowledge in
the domain of sensors devices and objects to interconnects \cite{midgar}.

\subsection{More challenges}
\label{sec:more_challenges}

Many more challenges exist in the area of \gls{IoT} development, they are not
going to be treated in this thesis because they are away from the main process
of simplification we want to bring with \gls{APDL}. The following list mention
another challenges that may be interresting to treat in future work.

\begin{itemize}
\item Engineering unit : Farenheit/Celsius, foot/meter, an so on...
\item Various network protocols : the device is connected with various protocols
  like ZigBee, MQTT, ...
\item Reliability, the device could be instantly checked and have some error
  handling application level.
\item Data description, we need to annotate the produced data (raw data are a
  problem).
\item Various security problems:
  \begin{itemize}
  \item Physical security.
  \item Data exchange security.
  \item Cloud storage security.
  \item Update and patch.
  \end{itemize}
\item Data curation and data brokering: device could produce a huge ammount of
  data
\item Data description: metadata to describe raw-data
\end{itemize}

\section{Programming languages for the Internet of Things}
\label{sec:pl_for_iot}

Using a programming language to create \gls{IoT} systems offer full control to
the developer in them of device design, conception, management and development.
In other hands, the developer has to do almost everything by itself. As mention
in \ref{sec:iot_challenge_optimisation}, the big majority of the embedded
devices does not suport an \gls{OS}.

\subsection{C/C++}
\label{subsec:cc++}

C and C++ are available for compilation on almost every platform, it’s a
hardware focused language which is not too complex and offers great speed and
memory management. The very big advantages is that C and C++ are compiled, they
don’t need a runtime management and compiled binaries are very small for limited
memory devices. From another point of view, the programmer needs to do almost
everything by itself: memory management, pointer arithmetic error handling
and, if we use C, there is no standard library for data structure and algorithm.

Notice that the Arduino language is a subset of C and owns the same advantages
and problems with memory management.

Therefore, C and C++ are really good when it comes to a small device with very
little memory and when no operating system are available.

\subsection{Python}
\label{subsec:python}

Python is a programming language created in 1990 by Guido van Rossum, it’s an
object-oriented, imperative and interpreted one. Because Python is interpreted,
the language needs to run with a virtual machine.

By the way, a tool named Cython\cite{behnel2010cython} is able to compile, after
some work, a Python code to a C executable, which removes the interpreter and
virtual machine needed at the origin.

We can go further, some embedded system for IOT does not run an operating system
(the Arduino for example) and Cython need to have access to some operating
system features. But there is some work about the possibility to run Python
without any OS\cite{jakeedge2015}.

On the other hand, Python owns a great community with tons of tutorials and library
for the IOT programming and especially with the Raspberry PI.

\subsection{Java}
\label{subsec:java}

C and C++ are great programming language for the IOT, it becomes easier to manage
and control the hardware with that language, but they are hardware specific
too. We can’t write a program and run it into every possible hardware. Maybe a
device is running on Linux and another one is running on Windows. The probably
best answer to this problem is Java\cite{Waher:2015:LIT:2789499}, which supports
multiple hardware design through his virtual machine if the IOT project run
needs to run on different platforms.

\subsection{Another Programming Languages}
\label{subsec:other-prog-lang}

There are many programming languages available for IOT development, we don’t want
to describe them all but we want a small overview of the field for the
development of the DSL. We could talk a lot about language like Go, Rust and
Javascript, which could be very good for a specific part of an IOT project.

We don’t need to know all the usable programming language for the IOT, but we
need a target for our DSL and those languages advantages and disadvantages
would be considering for further work.

\section{Domain specific languages for the Internet of Things}
\label{sec:dsl_for_iot}

Research and work has been done during the past year on the domain specific
language for the Internet of Things. Programming for the IOT could be really
anoying for the software developer who are not familliar with the hardware oriented
concept or for the electrical enginner which are not software oriented. We need
to know both soft and hard part of the project in order to realise it.

We present an overview of some research and technologies around the DSL created
for the IOT. Notice that when the DSL is a visual one, we call it a VDSML :
Visual Domain Specific Modeling Language.

\section{Node-red}
\label{sec:node-red}

Node-red\cite{node-red} is a tool, develop with Node.js and browser focused, for
wiring the internet of things together. It is a purely VDSML without, practicly,
no code to write. Node-red has compatibility with a lot of popular platforms\cite{node-red} :
\begin{itemize}
\item Raspberry Pi
\item BeagleBone Black
\item Android
\item Arduino
\item Docker
\item Amazon AWS
\item IBM Bluemix
\item Microsoft Azure
\end{itemize}

But we can go further, Node-red provides an editor, builtin function, a full
compatibility with Javascript and the Node.js packages and finally a way to save
and share graph information of the project with JSON.

Node-red is capable of simply create a complete pipeline from our diagram with
just a mouse click. Using block for programming and connect some basic features
do not require a high knowledge level.

\section{DSL-4-IoT}
\label{sec:dsl-4-iot}

TODO

\section{DiaSuite}
\label{sec:diasuite}

TODO
\section{PervML}
\label{sec:pervml}

TODO
\section{OpenHab}
\label{sec:openhab}

TODO
\section{LogicIOT}
\label{sec:logic-iot}

TODO
\section{ArduinoML}
\label{sec:arduino-ml}

TODO

%%% Local Variables:
%%% mode: latex
%%% TeX-master: "../thesis"
%%% End:
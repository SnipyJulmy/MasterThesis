\newChapter{Programming Language for the Internet of Things}
\label{cha:proglang-for-iot}

Using a programming language to create IOT Applications offer full control to
the developer, but it comes to the handling of the challenge too. By the way, some
language needs to embed interpreter and/or virtual machine in order to work. This
adds some additional memory, power and processor use.

\section{C/C++}
\label{subsec:cc++}

C and C++ are available for compilation on almost every platform, it’s a
hardware focused language which is not too complex and offers great speed and
memory management. The very big advantages is that C and C++ are compiled, they
don’t need a runtime management and compiled binaries are very small for limited
memory devices. From another point of view, the programmer needs to do almost
everything by itself: memory management, pointer arithmetic error handling
and, if we use C, there is no standard library for data structure and algorithm.

Notice that the Arduino language is a subset of C and owns the same advantages
and problems with memory management.

Therefore, C and C++ are really good when it comes to a small device with very
little memory and when no operating system are available.

\section{Python}
\label{subsec:python}

Python is a programming language created in 1990 by Guido van Rossum, it’s an
object-oriented, imperative and interpreted one. Because Python is interpreted,
the language needs to run with a virtual machine.

By the way, a tool named Cython\cite{behnel2010cython} is able to compile, after
some work, a Python code to a C executable, which removes the interpreter and
virtual machine needed at the origin.

We can go further, some embedded system for IOT does not run an operating system
(the Arduino for example) and Cython need to have access to some operating
system features. But there is some work about the possibility to run Python
without any OS\cite{jakeedge2015}.

On the other hand, Python owns a great community with tons of tutorials and library
for the IOT programming and especially with the Raspberry PI.

\section{Java}
\label{subsec:java}

C and C++ are great programming language for the IOT, it becomes easier to manage
and control the hardware with that language, but they are hardware specific
too. We can’t write a program and run it into every possible hardware. Maybe a
device is running on Linux and another one is running on Windows. The probably
best answer to this problem is Java\cite{Waher:2015:LIT:2789499}, which supports
multiple hardware design through his virtual machine if the IOT project run
needs to run on different platforms.

\section{Another Programming Languages}
\label{subsec:other-prog-lang}

There are many programming languages available for IOT development, we don’t want
to describe them all but we want a small overview of the field for the
development of the DSL. We could talk a lot about language like Go, Rust and
Javascript, which could be very good for a specific part of an IOT project.

We don’t need to know all the usable programming language for the IOT, but we
need a target for our DSL and those languages advantages and disadvantages
would be considering for further work.

%%% Local Variables:
%%% mode: latex
%%% TeX-master: "../thesis"
%%% End:
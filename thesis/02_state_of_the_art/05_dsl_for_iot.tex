\newChapter{Domain specific language for the Internet of Things}
\label{cha:dsl-for-iot}

Research and work has been done during the past year on the domain specific
language for the Internet of Things. Programming for the IOT could be really
anoying for the software developer who are not familliar with the hardware oriented
concept or for the electrical enginner which are not software oriented. We need
to know both soft and hard part of the project in order to realise it.

We present an overview of some research and technologies around the DSL created
for the IOT. Notice that when the DSL is a visual one, we call it a VDSML :
Visual Domain Specific Modeling Language.

\section{Node-red}
\label{sec:node-red}

Node-red\cite{node-red} is a tool, develop with Node.js and browser focused, for
wiring the internet of things together. It is a purely VDSML without, practicly,
no code to write. Node-red has compatibility with a lot of popular platforms\cite{node-red} :
\begin{itemize}
\item Raspberry Pi
\item BeagleBone Black
\item Android
\item Arduino
\item Docker
\item Amazon AWS
\item IBM Bluemix
\item Microsoft Azure
\end{itemize}

But we can go further, Node-red provides an editor, builtin function, a full
compatibility with Javascript and the Node.js packages and finally a way to save
and share graph information of the project with JSON.

Node-red is capable of simply create a complete pipeline from our diagram with
just a mouse click. Using block for programming and connect some basic features
do not require a high knowledge level.

\section{DSL-4-IoT}
\label{sec:dsl-4-iot}

TODO

\section{DiaSuite}
\label{sec:diasuite}

TODO
\section{PervML}
\label{sec:pervml}

TODO
\section{OpenHab}
\label{sec:openhab}

TODO
\section{LogicIOT}
\label{sec:logic-iot}

TODO
\section{ArduinoML}
\label{sec:arduino-ml}

TODO
\section{Internet of Things programming challenge}
\label{sec:iot-prog-challenge}

TODO

%%% Local Variables:
%%% mode: latex
%%% TeX-master: "../thesis"
%%% End:
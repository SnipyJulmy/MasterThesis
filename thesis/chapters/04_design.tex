\newChapter{Design}{chap:dsl_design}

\section{Introduction}
\label{sec:design_intro}

This chapter will present the design of the \gls{APDL} \gls{DSL}. Firstly, we
set out the domain of interest and some kind of a generalisation for the embedded
code we produce. The domain of interest regroups all the relevant information
related to the \gls{DSL} we create. This chapter will aslo present some general parts
of the embedded systems we want to generate. As a reminder, we want to generate
the code for some devices that are going to recover some data through
sensors, apply transformation on it and send them through a communication network
like a serial port.

The second part is about the fragmentation of \gls{APDL}. We are going to
present why \gls{APDL} is fragmented into multiple \gls{DSL} or languages and explain
the advantages and disadvantages of this method. We would also introduce the
PlatformIO system for embedded devices.

Finally, we present the final design chosen for \gls{APDL} \gls{DSL} and the
explanation on the choice of making an external \gls{DSL}.

\section{Domain of interest}
\label{sec:design_domain_of_interest}

According to \cite{little_languages_little_maintenance}, the development phase
of a \gls{DSL} requires a thorough understanding of the underlying domain. The
recommendation is to follow the next seven points\cite{little_languages_little_maintenance}.

\begin{itemize}
\item Identify problem domain of interest.
\item Gather all relevant knowledge in this domain.
\item Cluster this knowledge in a handful of semantic notions and operations on them.
\item Construct a library that implements the semantic notions.
\item Design a DSL that concisely describes applications in the domain.
\item Design and implement a compiler that translates DSL programs to a sequence of library calls.
\item Write DSL programs for all desired applications and compile them.
\end{itemize}

The point ``Identify problem domain of interest'' is treated in chapter
\ref{cha:iot_challenge}. In the next section, we will gather the relevant information on our
specific domain.

\section{Generalisation of the Embedded Framework}
\label{sec:generalisation_framework}

``What do we want ?''

That's the first questions to ask when we start analysing the domain's problems.
As an answer, we suggest the following :

{ \Large \textit{``We want to simply design a system, depending on a specific
    device, that is able of recovering some input data, transform and
    send them through a network.''}
}

Note that the previous sentence doesn't care about what's happening to the
transferred data, including handling, storage and visualisation. This part is
treated in chapter \ref{chap:apdl_ecosystem}.

We are going to take two examples, one with the Arduino Framework and one with
the Mbed SDK, and try to generalise the concept between each other. The listing
\ref{lst:arduino_generalisation} shows the Arduino code, and the listing
\ref{lst:mbed_generalisation} is showing the Mbed ones. The two codes have the
same goal :

\begin{itemize}
\item Recover two analogical inputs, the temperature and the luminosity.
\item Transform the temperature with a function.
\item Send the data each second.
\end{itemize}

\begin{listing}[H]
  \centering
\begin{scalacode}
#include "Timer.h"

Timer timer;

int pinLum = 1;
int pinTemp = 0;

float decode_temperature(int x){
  int B = 3975;
  float resistance = ((((float)(1023 - x)) * 1000) / x);
  float temperature = ((1 / ((log((resistance / 1000)) / B) + (1 / 298.15))) - 273.15);
  return temperature;
}

void sendLum(){
  int data = analogRead(pinLum);
  byte * b = (byte *) &data;
  Serial.write(b,4);
}

void sendTemperature(){
  int rawData = analogRead(pinTemp);
  float data = decode_temperature(rawData);
  byte * b = (byte *) &data;
  Serial.write(b,4);
}

void loop() {
  timer.update();
}

void setup() {
  Serial.begin(9600);
  timer.every(1000,sendLum);
  timer.every(1000,sendTemperature);
}
\end{scalacode}
  \caption[Arduino code for a simple data recovering]{Implementation of a simple
data recovering device with the Arduino Framework. The sampling is achieved
using a Timer and callbacks. The callback's methods are sending the data through a
serial network.}
  \label{lst:arduino_generalisation}
\end{listing}

\begin{listing}[H]
  \centering
\begin{scalacode}
#include <mbed.h>

Ticker ticker;

Serial pc(USBTX, USBRX);

AnalogIn rawTemp(PC_1);
AnalogIn lum(PC_3);

 // Transform tf
float tf (int x){
  int B = 3975;
  float resistance = ((((float)(1023 - x)) * 1000) / x);
  float temperature = ((1 / ((log((resistance / 1000)) / B) + (1 / 298.15))) - 273.15);
  return temperature;
}

void sendLum(){
  float data = lum.read();
  uint8_t * b = (uint8_t *) &data;
  unsigned int size = 0;
  while(size < sizeof(b)) {
      pc.putc(b[size++]);
  }
}

void sendTemperature(){
  float data = rawTemp.read();
  uint8_t * b = (uint8_t *) &data;
  unsigned int size = 0;
  while(size < sizeof(b)) {
    pc.putc(b[size++]);
  }
}

int main(void) {
  pc.baud(9600);
  ticker.attach(&sendLum,1.0);
  ticker.attach(&sendTemperature,1.0);
  while(1);
  return 0;
}
\end{scalacode}
  \caption[Mbed code for a simple data recovering]{Implementation of a simple
data recovering device with the Mbed Framework. The sampling is achieved
using a Ticker and callbacks. The callback's methods are sending the data through a
serial network.}
  \label{lst:mbed_generalisation}
\end{listing}

The two codes \ref{lst:arduino_generalisation} and \ref{lst:mbed_generalisation}
are very close to each other. We can recognise some key elements :

\textbf{An input} is represented by all the elements necessary to recover the
data from a sensor through specific pin. For Arduino, it's the code
\arduinoinline{analogRead(pinId)} which provides the data from the sensor. For
Mbed, it's the object \cppinline{AnalogIn} with the method \cppinline{read()}.
An input could be represented as a function $f : () \rightarrow A$where $A$ is
the type of the data we get when using the input.

\textbf{The sampling} indicates, in the software level, when the device needs
to recover the sensor data. The two main kinds of sampling are ``by update''
and ``by frequencies''. The sampling by update is presented in
listing \ref{lst:arduino_sampling_update}, the device sends the data only if it has
changed. In the sampling by frequencies showed in listing
\ref{lst:arduino_sampling_frequences}, the device sends the data on each
specified interval.

\begin{listing}[H]
  \centering
\begin{scalacode}
void setup(){
  timer.every(1000,byFrequences);
}

void byFrequences(){
  // Get data
  float data = tf(analogRead(1));
  // As a byte array…
  byte * b = (byte *) &data;
  // Send data
  Serial.write(b,4);
}
\end{scalacode}
  \caption[Sampling by frequencies implemented with an Arduino]{Sampling by update
    implemented with an Arduino. The device sends the value in each time interval.}
  \label{lst:arduino_sampling_update}
\end{listing}

\begin{listing}[H]
  \centering
\begin{scalacode}
void setup(){
  timer.every(1000,serial_lum);
}

void byUpdate(){
  // Get data
  float data = tf(tf(analogRead(1)));
  if(data != last_serial_temp2) {
    // As a byte array…
    byte * b = (byte *) &data;
    // Send data
    Serial.write(b,4);
  }
  last_serial_temp2 = data;
}
\end{scalacode}
  \caption[Sampling by update implemented with an Arduino]{Sampling by update
    implemented with an Arduino. Before sending the value, we test if it
    changes or not.}
  \label{lst:arduino_sampling_frequences}
\end{listing}

\textbf{A transformation} is a function $f : A \rightarrow B$ with an input
and an output, and is independent from the languages or framework.
The only important part of a transformation is its semantic. Depending on the
language, a transformation is not implemented in the same way but owns the same
properties. This concept gives the opportunity to design a higher level
language and then compile it into an embedded one.

A \textbf{device} is the hardware in which we are going to run the previous elements
(inputs, transformations, and so on…). The device depends on the hardware and
the software, sometimes it comes with a framework like mentioned
in \ref{sec:framework_for_embeded_dev}.

The last generalisation we made is called the ``Acquire and Process lifecycle''.
The lifecycle is represented by two steps in the life of an embedded device : the
setup and the loop. The setup is the first operation made by the device, like
configuration, calibration, and so on. The loop is the set of operations
executed permanentely until we stop the device.

\section{Multiple Domain Specific Languages}
\label{sec:multiple_dsl}

According to \ref{sec:design_domain_of_interest}, we are going to cluster the acquired
knowledge in a handful of semantic notions with operations on them and construct
a library with a language that implements those notions
\cite{little_languages_little_maintenance}.

From the user's point of view, we want to describe several devices that gather
inputs, apply some transformations and send them through a network with a
specific sampling. Those notions are translated into the following operations :

\begin{itemize}
\item Create a device.
\item Link some inputs with a device.
\item Create a transformation function.
\item Link an input with a transformation.
\item Send an input through a network with a sampling.
\item Specify the kind of sampling.
\end{itemize}

The problem is that we can't specify everything with a unique \gls{DSL}. The
transformation's implementation is not part of the devices and input
specifications. That's why we would have multiple \gls{DSL} in our ecosystem.
For example, if we want to add a way to specify the storage and the
visualisation of the gathered data, doing it inside the declarative \gls{DSL}
could create some inconsistency and behaviour misunderstanding.

\section{PlatformIO}
\label{sec:platformio}

PlatformIO\cite{Ivan2017} is an open source ecosystem for IoT development. It
provides a cross-platform IDE and unifies debuggers with remote unit testing and
firmware update\cite{Ivan2017}.

Such a build system is made for the kind of project we are doing with
\gls{APDL}. Each board supported by the PlatformIO is represented by a unique
identifier and the build systems provide an automatic downloading of libraries
and dependencies through a configuration file.

PlatformIO is used as a back end for \gls{APDL}, so only the PlatformIO
supported boards are available with \gls{APDL}.

\section{Design of the APDL's Domain Specific Languages}
\label{sec:design_apdl_dsls}

The goal of \gls{APDL} is to provide a simple way of declaring sensors oriented
systems. The design language needs to be simple and easy to read. Some highly
verbose languages are not easy to read for a human. We want to quickly
understand the purpose of a declared system.

The listing \ref{lst:apdlcode_all} is showing all the concepts discussed in
\ref{sec:multiple_dsl}. We can easily understand the purpose of the system :
gather two inputs, one on pin $1$ and one on pin $0$. Then we transform one and
send both through a serial network. One with a sampling of $1$ second and the
other is sent when it changes.

\begin{listing}[H]
  \centering
\begin{apdlcode}
@device arduino1 {
    id = uno
    framework = arduino
    @input rawTemp analogInput 1
    @input lum analogInput 0
    @input temp tf rawTemp
    @serial lum each 1 s
    @serial temp update
}
\end{apdlcode}
  \caption[Declaration of a device with the \gls{APDL} \gls{DSL}]{Declaration of
  a device using the \gls{APDL} \gls{DSL}. Everything is purely declarative, we
  never indicate how to do it.}
  \label{lst:apdlcode_all}
\end{listing}

The device declaration owns several information, each point indicates the
corresponding APDL code too :
\begin{itemize}
\item An identifier for the device, after the \apdlinline{@device} keyword.
\item A key-value pair, named \apdlinline{id}, which is the corresponding
  PlatformIO id of the boards\cite{Ivan2017}.
\item A key-value pair, named \apdlinline{framework}, which is corresponding to
  the development framework of the board, also available on PlatformIO\cite{Ivan2017}.
\item Some inputs, each one specified with an identifier, the type of the input
  and the arguments for the type.
\item Some serial, each one specified with an identifier and the sampling method
  and values.
\end{itemize}

\section{The APDL-Transform Domain Specific Language}
\label{sec:transformation_dsl}

The \gls{TF} language is another \gls{DSL} whose goal is to provide a high-level way of
coding a transformation. As explained in \ref{sec:generalisation_framework}, the
advantage to design a high-level language for transformation is the capability
of using a simple language and then to compile it to the desired platform.

The syntax of the \gls{TF} is quite similar to Scala and gives the traditional
concept for the user. The listing \ref{lst:tf-example} shows an example of a
transformation. This transformation is used to convert the resistance of the
sensor in Celsius.

\begin{listing}[H]
  \centering
\begin{apdlcode}
@define transform def tf (x:int) -> float {
    val B : int = 3975
    val resistance : float = (float)(1023 - x) * 1000 / x
    val temperature : float = 1 / (log(resistance/1000) / B + 1 / 298.15) - 273.15
    return temperature
}
\end{apdlcode}
  \caption[APDL transformation implement with the \gls{TF}
  language]{Implementation of a transformation function using the \gls{TF}
    language. The syntax is quite similar to Scala and gives the opportunity to
    write a high level code for embedded platform.}
  \label{lst:tf-example}
\end{listing}

The \gls{TF} language is fully integrated with \gls{APDL}, so we can modify the
language and extend it as much as we want. For the moment, the language offers
the following concept :

\begin{itemize}
\item Addition, subtraction, multiplication and division.
\item C-like type casting.
\item Variable and constant declarations.
\item Array creating and access.
\item Function definition and call.
\item Boolean expression.
\item Comparison operators.
\item Loops.
\item Flow-control structure : If-then-else, return, break and continue.
\item C-like type hierarchy.
\end{itemize}

Basically, all the implementation that comes after \apdlinline{@define transform}
is the declaration of a function with \gls{TF}. An introduction to the \gls{TF}
language is available in appendix \ref{app:tf-getting-started}.

\section{Extending the User Possibilities}
\label{sec:extending_user_possibilities}

In the section~\ref{sec:transformation_dsl} we introduce the keyword
\apdlinline{@define}. This keyword provides to the user the possibility to define
a new entity when designing a \gls{APDL} project. There are three different
possible kinds of definitions the user can create : transformations, inputs and
components. We already saw the transformation definition in
section~\ref{sec:transformation_dsl}.

\subsection{Defining New Inputs}
\label{sec:defining_new_input}

An input entity represents the way to recover the sensor information on a
device. For example, we would take the \apdlinline{analogInput} seen in
listing \ref{lst:apdlcode_all}. An analogical input is present on a lot of
platforms. For our example, we would take in consideration only the Arduino
platform\cite{ArduinoSoftware2017} and the Mbed platform\cite{ARMmbed}.

The analogical input is present in both frameworks, but its implementation and
usage for the user are quite different in both. In order to cover all kinds of
implementations for an input concept, we need to provide a general way to define
inputs. Such a way is shown in listing \ref{lst:define_analoginput}.

\begin{listing}[H]
  \centering
\begin{apdlcode}
@define input analogInput pin:str {
    @gen mbed {
        global = "AnalogIn @id(@pin);"
        setup = ""
        loop = ""
        expr = "@id.read()"
        type = float
    }
    @gen arduino {
        global = ""
        setup = ""
        loop = ""
        expr = "analogRead(@pin)"
        type = int
    }
}
\end{apdlcode}
  \caption[Definition of an analogical input using \gls{APDL}]{Definition of an
analogical input using \gls{APDL}. The generic part for any kind of frameworks is
present between the ``input'' identifier and the opening brackets. We could
define the identifier for the input and some arguments. Then we could write any
code that would be generated for the specified platform at the compile time.}
  \label{lst:define_analoginput}
\end{listing}

Listing \ref{lst:define_analoginput} also introduces the \apdlinline{@gen}
keyword. This keyword allows the user to define how to use the inputs in a
specified framework by giving some information in terms of key-value pairs :
global, setup, loop, expr and type. When we generalise an embedded device
implementation, we are bound to those concepts as shown in listings
\ref{lst:gen_expr_syntax_arduino} and \ref{lst:gen_expr_syntax_mbed}.

The ``setup'' and ``loop'' values are corresponding to their eponym discussed in
the section~\ref{sec:generalisation_framework}, the ``global'' value represents what all the
users want to put in the global scope, like global variables, function
definitions or constant. Finally, the ``expr'' and ``type'' values indicate to the compile how
to recover the input value and its type.

\begin{listing}[H]
  \centering
\begin{arduinocode}
/* Global */

void setup(){
  /* Setup */
}

void loop(){
  /* loop */

  /* type */ data = /* expr */ ;
}
\end{arduinocode}
  \caption[Generalisation of an embedded device lifecycle with
Arduino]{Generalisation of an embedded device lifecycle with the Arduino
framework. The framework already provides the ``loop'' and ``setup'' function.
This example also shows the result of the input definition for Arduino.}
  \label{lst:gen_expr_syntax_arduino}
\end{listing}

\begin{listing}[H]
  \centering
\begin{cppcode}
#include <mbed.h>

/* Global */

int main(void){
  /* Setup */

  while(1){
    /* Loop */
    
    /* type */ data = /* expr */ ;
  }
}
\end{cppcode}
  \caption[Generalisation of an embedded device lifecycle with
Mbed]{Generalisation of an embedded device lifecycle with the Mbed framework.
The framework does not provide the ``loop'' and ``setup'' abstraction like
Arduino does, so we need to simulate them. This example also shows the result of
the input definition for Arduino.}
  \label{lst:gen_expr_syntax_mbed}
\end{listing}

The strings present in the values for the \apdlinline{@gen} definition are a kind
of interpolated strings and have access to the parameters through the macro
\apdlinline{@parameterIdentifier}. \gls{APDL} also provides the macro
\apdlinline{@id} which is a unique identifier generated by the compiler.

Now, admitting we generate the ``analogInput'' definition for Mbed, we would
obtain the result shown in listing~\ref{lst:analogInput_result_example}. We
simply generate the defined input with the information about the generation
given by the user.

\begin{listing}[H]
  \centering
\begin{cppcode}
#include <mbed.h>

AnalogIn x_1(A1);

int main(void){

  while(1){
    float data = x_1.read() ;
  }
}
\end{cppcode}
  \caption[Generation of an input definition for Mbed]{Generation of an input
definition for the Mbed framework. All we do is just replacing the arguments
for the definition and then generate the code at the specified points showed
in listing \ref{lst:gen_expr_syntax_arduino}. We admit that the ``pin''
parameter value is equal to $1$. The \apdlinline{@id} value is generated by the
compiler.}
  \label{lst:analogInput_result_example}
\end{listing}

By using this kind of definition and generation, we can write any kind of inputs
on any platforms. By offering a general description for the generation, the user
could also define some inputs that require a library. In this case, the user can include
the library by using the global value.

\subsection{Defining New Components}
\label{sec:defining_component}

We could go further than the input's definition mechanism by offering to the
user the possibility to define components. A component could be seen as a
black box, with some inputs and an output. An example of a component design with
\gls{APDL} is shown in listing \ref{lst:apdlcode_component_example}.

\begin{listing}[H]
  \centering
\begin{apdlcode}
@define component simpleOperator op:str {
    @in x:int y:int
    @out int
    @gen mbed {
        global = ""
        setup = ""
        loop = ""
        expr = "@x @op @y"
    }
    @gen arduino {
        global = ""
        setup = ""
        loop = ""
        expr = "@x @op @y"
    }
}
\end{apdlcode}
  \caption[Definition of a component with \gls{APDL}]{Definition of a component
with \gls{APDL}. A component's definition is quite similar to an input
definition except that the output type is not depending on the framework. A
component definition is a kind of macro system for \gls{APDL}. Another
difference is the generation. An input is used only for recovering the input and
a component is generated as a function.}
  \label{lst:apdlcode_component_example}
\end{listing}

At the generation time, a component is described as a function $f :
(A_0,A_1,...A_n) \rightarrow B$ where $A_0,...,A_n$ ,are the parameter's types of
the functions and $B$ is the return type.

Using a component is almost the same as using an input. We just have to
specify the type of the component and give the exact number of arguments.
The arguments are, in order, the device parameters and then the input parameters :

\apdlinline{@input componentOutput simpleOperator + input1 input2}

The generated code is shown in listing \ref{lst:component_generated}.

When we use this implementation and the definition of the component defined in
listing \ref{lst:apdlcode_component_example}, we obtain the code from listing
\ref{lst:component_generated}. All the parameters are interpolated at the
compile time.

\begin{listing}[H]
  \centering
\begin{cppcode}
int component_simpleOperator_tempLum(int x,int y) {
  return x + y;
}
\end{cppcode}
  \caption[Generated code for an \gls{APDL}'s component]{Generated code from the
component defined in listing \ref{lst:apdlcode_component_example}. A component is
represented as a function, and it is parameterised with its arguments.}
  \label{lst:component_generated}
\end{listing}

\section{Summary}
\label{sec:design_summary}

In this chapter, we have presented the design process and choices on the multiple
\gls{APDL}'s \gls{DSL}s. We set out the \gls{APDL}'s domain of interest by
introducing the development process of a \gls{DSL}
by~\cite{little_languages_little_maintenance}. Then we suggested a
generalisation in section~\ref{sec:generalisation_framework} for any sensor-oriented
projects, independently from any platforms, by introducing the concept of : input, devices,
transformation, sampling and the \gls{APDL} lifecycle with the steps called
``setup'' and ``loop''.

After, we discovered in section~\ref{sec:multiple_dsl} why we need multiple \gls{DSL} for \gls{APDL} and argued that
implementing everything with a single \gls{DSL} could create some inconsistency
and behaviour misunderstanding.

Finally, we set up the design of all the \gls{DSL} and concepts created with
\gls{APDL} : the \gls{APDL} \gls{DSL}, the \gls{TF}
language for transformation and the possibility of defining new
inputs in section~\ref{sec:defining_new_input} and new
components in section~\ref{sec:defining_component}.

%%% Local Variables:
%%% mode: latex
%%% TeX-master: "../thesis"
%%% End:
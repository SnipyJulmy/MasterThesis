\newChapter{Validation and Results}{chap:dsl_validation}

\section{Introduction}
\label{sec:validation_intro}

This part will present the tests, and results obtained with \gls{APDL}. We will
introduce a technique called property-based testing, in which we don't care
about the inputs of the test, but about the properties of an entity. Then, we
will introduce the implementation of the inputs generators for the tests
execution.

Afterwards, we will talk about the validation of \gls{APDL} towards the other
\gls{DSL} mentioned in chapter \ref{cha:iot_challenge}. We will also indicate why
it is hard to validate a \gls{DSL} in specific area.

Finally, we are going to discuss about the novelties suggested by \gls{APDL} like
as the generalisation of the frameworks for embedded development.

\section{Testing the parser}
\label{sec:testing_the_parser}

A big part of testing is the generation of concrete inputs. It's almost
impossible for a single human to create any kind of possible inputs for a parser.
That's why, for the project purpose, we are using a technique called
property-based testing.

Property-based testing is a kind of testing that allows the user to specify
assertions about logical properties which a test function should fulfil. In our case,
what the \gls{APDL} parser should parse and what it should not parse.

ScalaCheck~\cite{RickardNilsson} is a library that offers property-based testing
and also allows the user to create his own generators. For \gls{APDL}, we have
created a set of generators, which are capable of creating any kind of possible
valid \gls{APDL} projects. The full implementation of those generators is
available in appendix \ref{app:generators_property_based_testing}.

We will just take an example in order to understand what we have done to test
the parser. The following generator is able to generate an expression for the
\gls{TF} language :

\begin{inlinescala}
def genExpr: Gen[Expr] = genExprInner(maxExprSize)

private def genExprInner(depth: Int): Gen[Expr] = {
  if (depth == 0) genExprTerminal
  else {
    val nextDepth = depth - 1
    Gen.oneOf(
      genAdd(nextDepth),genMul(nextDepth),genDiv(nextDepth),
      genSub(nextDepth),genCast(nextDepth),genOr(nextDepth),
      genAdd(nextDepth),genNot(nextDepth),
      genSmaller(nextDepth),genSmallerEquals(nextDepth),
      genEquals(nextDepth),genNotEquals(nextDepth),
      genGreater(nextDepth),genGreaterEquals(nextDepth),
      genFunctionCall(nextDepth),genSymbol,
      genLiteral,genTrue,genFalse
    )
  }
}
\end{inlinescala}

In order to prevent the framework of generating huge expression which would cause
stack overflow, we use a parameter to indicate the maximal depth for
expression.

All of the functions contained in the \scalainline{Gen.oneOf} methods are
generators too and are mostly the same as

\begin{inlinescala}
def genAdd(depth: Int): Gen[Add] =
    for {
      e1 <- genExprInner(depth)
      e2 <- genExprInner(depth)
    } yield Add(e1, e2)
\end{inlinescala}

Once the generators are defined, we can implement a code generator that is
producing \gls{APDL} source code. The generated code allows us to parse it with
the \gls{APDL} parser and finally, we test if the two resulting \gls{AST} are equals.
The code generator for \gls{APDL} is available in appendix
\ref{app:apdl_code_generator_for_test}.

When we have both of those components, we could write tests about logical
assertion. Listing \ref{lst:test_expr} shows a property which asserts that any
generated expression has to be equal to the parsed, and generated code.

\begin{listing}[H]
  \centering
\begin{scalacode}
behavior of "The TransformApdlParser parser"

it should "correctly parse any well formed expr" in {
  val gen = new ApdlExprGenerators(5)
  val codeGen: TransformApdlBackendGenerators = new TransformApdlBackendGenerators {}

  check {
    forAll(gen.genExpr) { e =>
      val code = codeGen.toApdlCode(e)
      val ast = parse(code,tfExpr)
      ast == e
    }
  }
}
\end{scalacode}
  \caption[Property testing of an \gls{APDL} expression]{Property of an
\gls{APDL} expression. We assert that for all generated expressions, both of the
\gls{AST} have to be equal.}
  \label{lst:test_expr}
\end{listing}

We can't assume that our parser is correct even with those tests. It's possible
that we have made the same error twice, so the test will pass but the parser
isn't correct.

We could mention that a test has been created for the preprocessor. \gls{APDL} allows
the user to separate his projects into multiple files and use the
\scalainline{@include} keyword in a C style. The include commands are just
replaced by the contents of the specified file. The implementation of this test
is available in appendix \ref{app:includes_tests}.

\section{Testing the Code Generation}
\label{sec:testing_code_generation}

Testing the code generation is harder than testing the parser and
consumes much more time. Because we generate the code for different platforms,
we have to write many tests cases, and we can't use the
generators discussed in section \ref{sec:testing_the_parser}. We don't have a
parser for C source code that can create an \gls{APDL} \gls{AST}.

The only test we can make without consuming too much time is to write some
\gls{APDL} source code, generate it and assert that the generated code is
working on the expected goal. The sections~\ref{subsec:test_A}
to~\ref{subsec:test_C} present various APDL source codes and their corresponding
generated outputs, each test indicates if the semantics of the APDL source is
respected in the output. Each output has been runned on the specified device
and the semantics is respected.

\subsection{Test A}
\label{subsec:test_A}

\subsubsection*{Input}
\begin{apdlcode}
@device A {
    id = uno
    framework = arduino
    @input t analogInput 1
    @serial t update
}
\end{apdlcode}

\subsubsection*{Output}
\begin{arduinocode}
#include "Timer.h"

Timer t;

int last_serial_t;

void serial_t() {
    // Get data
    int data = analogRead(1);

    if(data != last_serial_t) {
        char buffer[1024];
        sprintf(buffer,"t : %d", (int)data);
        //sprintf(buffer,"t : %d", data);
        Serial.println(buffer);
    }
    last_serial_t = data;
}

void loop(){
    t.update();
}

void setup() {
    Serial.begin(9600);
    last_serial_t = analogRead(1);
    t.every(1000,serial_t);
}
\end{arduinocode}

\paragraph*{Semantic is respected : yes}
\newpage

\subsection{Test B}
\label{subsec:test_B}

\subsubsection*{Input}
\begin{apdlcode}
@device B {
    id = disco_f429zi
    framework = mbed
    @input t analogInput 1
    @input t1 tf t
    @input t2 tf t
    @serial t2 each 2 s
}

@define transform def tf (x:int) -> float {
    val B : int = 3975
    val resistance : float = (float)(1023 - x) * 1000 / x
    val temperature : float = 1 / (log(resistance/1000) / B + 1 / 298.15) - 273.15
    return temperature
}
\end{apdlcode}

\subsubsection*{Output}
\begin{cppcode}
#include <mbed.h>

Ticker ticker;

Serial pc(USBTX, USBRX);

AnalogIn temp_2(1);

float tf (int x) {
    int B = 3975;
    float resistance = ((((float)(1023 - x)) * 1000) / x);
    float temperature = ((1 / ((log((resistance / 1000)) / B) + (1 / 298.15))) - 273.15);
    return temperature;
}

void serial_t2() {
    // Get data
    float data = tf(temp_2.read());
    pc.printf("t2 : %f",data);
}

int main(void) {
    pc.baud(9600);
    // Setup
    ticker.attach(&serial_t2,2.0);

    // Loop
    while(1) {
    }

    return 0;
}
\end{cppcode}

\paragraph*{Semantic is respected : yes}
\newpage

\subsection{Test C}
\label{subsec:test_C}

\subsubsection*{Input}
\begin{apdlcode}
@device arduino1 {
    id = uno
    framework = arduino
    @input rawTemp analogInput 1
    @input lum analogInput 0
    @input temp tf rawTemp
    @input temp2 tf temp
    @input lumTemp simpleOperator + lum temp
    @input tempLum simpleOperator - temp lum
    @input lc loopCounter
    @serial lum each 1 s
    @serial temp each 1 s
    @serial temp2 update
    @serial lumTemp update
    @serial lc update
}

@define transform def tf (x:int) -> float {
    val B : int = 3975
    val resistance : float = (float)(1023 - x) * 1000 / x
    val temperature : float = 1 / (log(resistance/1000) / B + 1 / 298.15) - 273.15
    return temperature
}

@define component simpleOperator op:str {
    @in x:int y:int
    @out int
    @gen mbed {
        global = ""
        setup = ""
        loop = ""
        expr = "@x @op @y"
    }
    @gen arduino {
        global = ""
        setup = ""
        loop = ""
        expr = "@x @op @y"
    }
}

@define component loopCounter {
    @in
    @out int
    @gen arduino {
        global = "int a;"
        setup = "a = 0;"
        loop = "a = a + 1;"
        expr = "a"
  }
}
\end{apdlcode}
\newpage

\subsubsection*{Output}
\begin{arduinocode}
#include "Timer.h"

Timer t;

int a;
float last_serial_temp2;
int last_serial_lumTemp;
int last_serial_lc;

// Transform tf
float tf (int x) {
    int B = 3975;
    float resistance = ((((float)(1023 - x)) * 1000) / x);
    float temperature = ((1 / ((log((resistance / 1000)) / B) + (1 / 298.15))) - 273.15);
    return temperature;
}
// End transform tf

// Component simpleOperator
int component_simpleOperator_lumTemp(int x,int y) {
    return x + y;
}
// End of simpleOperator

// Component simpleOperator
int component_simpleOperator_tempLum(int x,int y) {
    return x - y;
}
// End of simpleOperator

// Component loopCounter
int component_loopCounter_lc() {
    return a;
}
// End of loopCounter

void serial_lum() {
    // Get data
    int data = analogRead(0);
    char buffer[1024];
    sprintf(buffer,"lum : %d", (int)data);
    //sprintf(buffer,"lum : %d", data);
    Serial.println(buffer);
}

void serial_temp() {
    // Get data
    float data = tf(analogRead(1));
    char buffer[1024];
    sprintf(buffer,"temp : %d", (int)data);
    //sprintf(buffer,"temp : %f", data);
    Serial.println(buffer);
}

void serial_temp2() {
    // Get data
    float data = tf(tf(analogRead(1)));

    if(data != last_serial_temp2) {
        char buffer[1024];
        sprintf(buffer,"temp2 : %d", (int)data);
        //sprintf(buffer,"temp2 : %f", data);
        Serial.println(buffer);
    }

    last_serial_temp2 = data;
}

void serial_lumTemp() {
    // Get data
    int data = component_simpleOperator_lumTemp(analogRead(0),tf(analogRead(1)));

    if(data != last_serial_lumTemp) {
        char buffer[1024];
        sprintf(buffer,"lumTemp : %d", (int)data);
        //sprintf(buffer,"lumTemp : %d", data);
        Serial.println(buffer);
    }

    last_serial_lumTemp = data;
}

void serial_lc() {
    // Get data
    int data = component_loopCounter_lc();

    if(data != last_serial_lc) {
        char buffer[1024];
        sprintf(buffer,"lc : %d", (int)data);
        //sprintf(buffer,"lc : %d", data);
        Serial.println(buffer);
    }

    last_serial_lc = data;
}

void loop() {
    t.update();
    a = a + 1;
}

void setup() {
    Serial.begin(9600);
    a = 0
        t.every(1000,serial_lum);
    t.every(1000,serial_temp);
    last_serial_temp2 = tf(tf(analogRead(1)));
    t.every(1000,serial_temp2);
    last_serial_lumTemp = component_simpleOperator_lumTemp(analogRead(0),tf(analogRead(1)));
    t.every(1000,serial_lumTemp);
    last_serial_lc = component_loopCounter_lc();
    t.every(1000,serial_lc);
}
\end{arduinocode}

\paragraph*{Semantic is respected : yes}
\newpage

\section{Results}
\label{sec:results}

Due to the time taken at the beginning of the project for trying to use
\gls{LMS} in order to implement \gls{APDL}, we didn't have enough time left
for a proper validation of the results. However, we have created multiple
\gls{APDL} projects using the \gls{DSL} and have obtained some good results.

\subsection{Code Generation}
\label{sec:res_code_generation}

The code generation is working for Arduino and for Mbed. PlatformIO's projects
are well generated and also used for library dependencies. PlatformIO enables us
to specify several libraries and the tool downloads them itself.

The results about the code generation are also part of the testing on the same
subject discussed in sections \ref{sec:testing_code_generation}. In order to
obtain some verifiable results, we have to use \gls{APDL} and collect feedback.

\subsection{APDL Versus the Rest of the World}
\label{sec:apdl_vs_world}

Comparing \gls{APDL} with the  \gls{DSL} analysed in
chapter~\ref{cha:iot_challenge} isn't trivial. We haven't found a
\gls{DSL} which is trying to do the same thing as ours. We are probably the only
one who are working on this kind of \gls{DSL} which is targeting low-level device for
a specific kind of work.

Between all the mentioned \gls{DSL} in chapter~\ref{cha:iot_challenge},
Node-Red\cite{node-red} is the most famous and used one. In contrast to \gls{APDL},
Node-red is acting on a high-level entity like Twitter, Email or MQTT messages.
There are no notions of low-level or embedded devices. Some extensions exist for
connecting Raspberry Pi with Node-red, but that's not the priority of the developers.

An idea to validate \gls{APDL} would be to create a set of
representative projects to implement with various kinds of \gls{DSL} for
\gls{IoT} and to compare the results. The issue with this validation is the time.
The necessary time to find people who already know the other technologies is
huge. In addition, as mentioned in chapter~\ref{cha:a-dsl}, the purpose of a
\gls{DSL} is to be domain-specific. Comparing multiple \gls{DSL} with different
purposes is probably not a good idea.

The last way to validate \gls{APDL} is to use it over the time and get the
feedback from the user in order to improve it. This brings us to one of the key
purposes of a \gls{DSL}. The biggest advantage of developing a \gls{DSL} is also
its biggest disadvantage : the domain-specific orientation. The domain-specific
constraint is very painful when comes time to validate the result
against another similar purpose product.

\section{APDL Contribution to the Internet of Things}
\label{sec:apdl_contribution}

\gls{APDL} is, by far, not finish yet. A lot of work needs to be done to
round off and improve the concept, but the foundations are here. With \gls{APDL},
we have shown that it is possible to design a very high definition language for
very low-level purpose. One step further, the language is purely declarative
except for the definition of new entities. A declarative language is, in
general, simpler to read than an imperative one.

Another concept is the generalisation of the embedded devices. Conceptually,
any \gls{IoT} device and associated framework could be generalised into an
entity which contains the following properties:

\begin{itemize}
\item An initialisation phase, which is the set of operations executed once,
  and called ``setup''.
\item A repeated loop phase, which is the set of operations executed at each
  interval of times, and called ``loop''.
\item Some input functions or entities, which are gathering the environment
  values into the software part.
\item Some transformation functions, which are manipulating the input values.
\item Some components, which contain what we can't generalise for any frameworks or devices.
\end{itemize}

Even if \gls{APDL} is not yet ready for the production, the established concept
could bring novelty to the \gls{IoT} world. The improvement of \gls{APDL} or
even the development of new languages, frameworks or \gls{DSL} could lead to a new
era of progress in \gls{IoT} development.

\section{Summary}
\label{sec:summary}

We have presented the validation and the results obtained with the \gls{APDL}
\gls{DSL}. We introduced the property-based testing techniques used to validate
the parsers. We generated random \gls{AST} which are converted into \gls{APDL}
code, before parsed by the parser and transformed into a new \gls{AST}.

Testing the code generation is a lot trickier and time consuming, a set of
tests have been created in order to obtain some kind of a validation.

Finally, we discussed about the novelties brought by \gls{APDL} into the
\gls{IoT} field. We saw that even if \gls{APDL} is not ready for the production,
the concept known from the development phase is relevant for the future
development of the \gls{IoT}.

%%% Local Variables:
%%% mode: latex
%%% TeX-master: "../thesis"
%%% End:
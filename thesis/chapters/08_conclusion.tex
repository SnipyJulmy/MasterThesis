\newChapter{Conclusion}{chap:conclusion}

\section{Summary}
\label{sec:summary}

Looking back to the challenge laid down at the beginning of this project, we
have shown with concrete examples how we can generalise the concept of
\gls{IoT} and embedded programming. The cases we have investigated,
generalisations of embedded programming, \gls{DSL} for the development of
specific \gls{IoT} projects and the conception of an entire ecosystem based on
\gls{APDL}, are promising and could lead to a new era of simplification and
development for the \gls{IoT} world.

In chapter~\ref{cha:a-dsl} and \ref{cha:iot_challenge}, we set our work's
foundations. The design and development of a \gls{DSL} is complex and the
specific domain of investigation, \gls{IoT} and embedded devices, is wide and involves different technologies.

Chapter~\ref{chap:dsl_design} deal with the design of the APDL \gls{DSL} and
explain the choices made during its construction, especially the fragmentation of
APDL into multiple DSL. We also showed the design of the \gls{TF} language and
argued why we need an additional programming language to describe
transformation. Finally, we suggested the proposition of generalising the
development of sensor-oriented projects. We introduced the concepts of inputs,
devices, transformations, samplings and APDL lifecycle.

In chapter~\ref{chap:dsl_implementation}, we illustrated and explained the
implementation of the APDL compiler. The compilation process shown in
figure~\ref{fig:apdl_compilation_process} decrypt perfectly the modelling of
the compiler. This chapter also notes that the compiler is by far not finish and
some first-class features and improvements have to be done before running APDL
into the production.

Chapter~\ref{chap:dsl_validation} try to validate the work done so far. Firstly,
we introduced the concept of property-based testing and its application to
the parser testing and validation. In a second time, we presented the generators used
to generate almost any kind of valid program for APDL. We also tried to present
why it's hard to validate a \gls{DSL} against another technology and suggested a
last way to validate APDL. At the end, we presented the contribution of APDL to
the \gls{IoT} world, and the properties hold by the generalisation of any
\gls{IoT} devices.

Finally, chapter~\ref{chap:apdl_ecosystem} explains the technological choices
made in order to use the APDL DSL inside a bigger system. Grafana and InfluxDB
are both free and easy to use for any people. At the end, we presented a complete
realisation of a simple \gls{IoT} project from scratch and proved that it's
possible to simplify the development process of \gls{IoT} devices.

In the following sections, we will present the future of the projects, and
the possible improvements and features to implements in APDL. We will end this
thesis by the acknowledgement for all the people involved in this work and with a
personal statement.

\section{Further Work}
\label{sec:further_work}

As said before, the APDL compiler is not ready for the production but the
foundations for future work are ready. APDL is an open source project, and its
usage is not limited in any way. Contributions are welcome from anybody.
The following list shows the improvements and features that could be added to APDL
:

\begin{itemize}
\item Verify the types during the compilation and a better type system in
  general.
\item Verify and prevents variable name collision in the same scope.
\item Allow the definition of a new framework with the APDL language in order to
  prevent compiler modification.
\item Implement a standard library and a library manager for APDL in APDL.
\item Validate the compiler by using APDL in larger projects and get feedback
  from the users.
\end{itemize}

\section{Acknowledgement}
\label{sec:acknowledgement}

I would like to sincerely thank the following people, without whom this project
would have been impossible to realise. Their works, support and advice have
been crucial during those four months of hard work.

\begin{itemize}
\item Professor Dr. Pierre-André Mudry, the initiator and supervisor of this
  thesis, for supporting me and providing the key advice to complete this
  project.
\item My mother, Manuela Julmy, for the support and advice she gave me
  during all this period.
\item My father, Roland Julmy, for having reviewed this document and encouraging
  me.
\item My uncle, Marcel Julmy, for having reviewed this document and encouraging
  me.
\item To all the people or organisation I forgot, many thanks !
\end{itemize}

\section{Personal statement}
\label{sec:personal_statement}

I have spent the last four months on this thesis, and working alone most of the
time could be really hard. As mentioned previously, the support I received during
this time was crucial for the success of this thesis, but also for me not to
be discouraged by the complexity of this task.

The results are promising and I am satisfied of the obtained results. I
believe that APDL could be used by different companies to simplify some internal
process. I see an application of this work in the growing fields of Data Science
and Machine Learning, in which the systems need more and more data to analyse.

Finally, working on this thesis open me to several domains and technologies that
I didn't know that much. I took the opportunity to improve my knowledge in
several fields such as embedded hardware and software, compiler construction or
IoT concepts. I enjoyed a lot doing this thesis and hope that all the accumulate
knowledge is going to be useful for any interested people.

Fribourg, the 29th of June of 2017

Sylvain Marcel Julmy

%%% Local Variables:
%%% mode: latex
%%% TeX-master: "../thesis"
%%% End:

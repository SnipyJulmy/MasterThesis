\newChapter{Internet of Things Development}
\label{cha:iot_challenge}

\section{Introduction}
\label{sec:iot_challenge_intro}

Internet of Things has become a huge area of research and development over the past few
years. With the emergence of data science and libraries for big data analysis
like \gls{Hadoop} and \gls{Spark}, the scientist wants more and more data to
analyse.

Recovering environment data with embedded hardware is present from the start of
the '60s\cite{Community2017}. In contrary to standard computer systems, embedded
systems and \gls{IoT} systems suffer of several challenges.

This chapter introduces the challenge of the \gls{IoT} development, from the device
itself to the software implementation. Then, we will set out the development of
\gls{IoT} projects using general programming languages, framework for embedded
development and \gls{DSL} for \gls{IoT} systems.

\section{Internet of Things challenges}
\label{sec:iot_challenges}

\subsection{Heterogeneity}
\label{sec:heterogeneity}

As the first challenge for the system development in \gls{IoT}, we should
mention the disparity of the devices. The heterogeneity of the objects does not
allow them to interact \cite{midgar}. In order to connect them, we need a
common interface like a central server or a common description language.

It exists various devices and development frameworks in the world of
embedded hardware: Arduino, Raspberry PI, Mbed, SPL, Simba, Teensy, and so
on. Each of them with specific properties and specific programming language.

\subsection{Optimisation}
\label{sec:iot_challenge_optimisation}

One big limitation for \gls{IoT} system development is the power management. It
directly affects the algorithms we can use and means that the solutions we
create would always be the power-optimised one\cite{Sneps-Sneppe2016}. In
general, the power management can be done by the operating, which is a luxury
to have in most parts of the available devices.

We have to mention in this context a technique named \gls{DPM}. The idea is to
shut down the devices when they aren't busy and turn them on when they
need to receive/send data. In fact, it's kind of a replacement for the
\gls{OS}. But using such a technique could cause latency and additional development.

\subsection{Connectivity}
\label{sec:connectivity}

In order to communicate together, and as previously announced in the section \ref{sec:heterogeneity}, \gls{IoT} devices need to send information through
a standard interface. Developing such interface requires experience and knowledge
in software development and programming language as well as a wide knowledge in
the domain of sensor devices and objects to interconnects. \cite{midgar}

\subsection{More Challenges}
\label{sec:more_challenges}

Many more challenges exist in the area of \gls{IoT} development, they will not be dealt
in this thesis because they are away from the main process
of simplification we want to bring with \gls{APDL}. The following list mentions
other challenges that may be interesting to be dealt in the future. 

\begin{itemize}
\item Engineering unit: Fahrenheit/Celsius, foot/metre, and so on…
\item Various network protocols: the device is connected with various protocols
like ZigBee, MQTT, and so one…
\item Reliability, the device could be instantly checked and have some error
handling application level.
\item Data description, we need to annotate the produced data (raw data are a
problem).
\item Various security problems:
  \begin{itemize}
  \item Physical security.
  \item Data exchange security.
  \item Cloud storage security.
  \item Updates and patches.
  \end{itemize}
\item Data curation and data brokering: a device may produce a huge amount of
data and we need to deal with.
\item Data description: metadata to describe raw data.
\end{itemize}

\section{Programming languages for the Internet of Things}
\label{sec:pl_for_iot}

Using a programming language to create \gls{IoT} systems offer full control to
the developer in terms of device design, conception, management and development.
In other hands, the developer has to do almost everything by himself. As mentioned
in \ref{sec:iot_challenge_optimisation}, the great majority of the embedded
devices does not support an \gls{OS}.

\subsection{General Programming Languages}
\label{sec:gen_pl}

Use a general programming language for \gls{IoT} development offers a lot to the
user, he could design and implement almost anything he wants. Some languages have
incorporated a library and/or functions in order to access the hardware directly so
anyone could develop an \gls{IoT} device.

\textbf{C and C++} are both a good choice for such a project. Compilation targets are
available for almost any platform and offer great speed and memory management.
The very big advantage is as they are compiled languages, they don't need a
runtime management and compiled binaries are very small for a device with a
limited memory. From another point of view, the programmer needs to do almost
everything by himself: memory management, pointer arithmetic error handling
and, if we use C, there is no standard library for data structure and algorithm.

\textbf{Python} is a language to mention as well, it has been created in 1990
by Guido van Rossum \cite{Rossum:1995:PRM:869369}. Instead of C and C++, Python
needs a virtual machine in order to run and that is not acceptable for a small
device. If the device we use owns an \gls{OS}, Python is one of the best choices
for the development. The language owns a great community with many tutorials
and libraries for the \gls{IoT} programming and especially with the Raspberry PI. We
could mention too that a tool named Cython\cite{behnel2010cython} is able to
compile, after some work, a Python code to a C executable, which removes the
interpreter and virtual machine needed at the origin. Another idea is to run
Python without any \gls{OS}\cite{jakeedge2015}.

\textbf{Java} is also a great choice if the development of the system involves
various devices with different platforms and when an \gls{OS} is available. When we use C
or C++, we need to clearly identify the hardware. When we have
multiple devices with various architectures, we have to code platform-specific
code for each of them and that involves poor scaleability and reutilisability.
Java offers to run on any platform if an \gls{OS} and a \gls{JVM} is available.

It exists a lot more programming language for \gls{IoT} development: Javascript,
Go, Rust, Parasail, B\#, Assembly, and so on… We mentioned C, C++, Java and
Python in order to expose the advantages and disadvantages to use a general
programming language for \gls{IoT} systems.

\subsection{Framework for embedded development}
\label{sec:framework_for_embeded_dev}

Over the past years, multiple frameworks for embedded systems appear and grew up.
That kind of framework simplify a lot the development process for embedded and
\gls{IoT} systems. We will explore in particular two of them: Arduino and
Mbed.

\textbf{Arduino} is an open-source platform based on easy-to-use hardware and
software\cite{Arduino2017}. Such an electronic card is capable of reading input
like sensors or button, as well as generating output through serial or LED. In
order to use the hardware, a programming language named Arduino as well and
based on Wiring\cite{Arduino2017} is provided. This language is kind of a
subset of C but with the addition of classes.

\textbf{Mbed} is an open-source operating system based on 32-bit ARM Cortex-M
micro-control\-lers\cite{Toulson:2012:FEE:2385440}. Mbed has been specially designed
for \gls{IoT} device design and development. The language used by the developer
is C++ with the addition of the Mbed SDK\cite{Toulson:2012:FEE:2385440}.

\section{Domain Specific Languages for the Internet of Things}
\label{sec:dsl_for_iot}

Research and work has been done during the past years on the \gls{DSL} for the
\gls{IoT}. As presented in section \ref{sec:iot_challenges}, \gls{IoT} projects may
involve a lot of different aspects, technologies and knowledge. From hardware
specification to network protocols passing by software design and
implementation, we are not going to present all the \gls{DSL} for the \gls{IoT}.
We present the ones that have been chosen for the proximity with this thesis purpose.

\textbf{Node-red} is a \gls{VDSML}, developed with Node.js and
Javascript whose goal is to wire together hardware devices, API and online
services\cite{node-red}. A flow based editor is provided in the browser and
the user is invited to create nodes, wire them, deploy and run the projects
with a maximum of simplicity\cite{Salihbegovic2015}. The runtime engine is built
on top of Node.js and make use of its event-driven and non-bloking model. This
makes it ideal to run on small hardware such as Arduino or
Raspberry\cite{node-red}. By the way, Node-red is by far the most well-known and
used \gls{DSL} in the area of \gls{IoT}.

\textbf{DiaSuite} is a tool suite, designed by the Inria, that uses a
software design approach to drive the development process\cite{Bertran2012}.
The tool is focused on the \gls{SCC} architecture. The \gls{SCC} architecture is
a pattern involved in a system like: building automation, application monitoring,
robotics, autonomic computing\cite{Taylor:2009:SAF:1538494}. The tool suite
includes a domain-specific design language, a compiler to Java code, a simulator
and a deployment framework. 

\textbf{DSL-4-IoT} is an Editor-Designer based on a high level \gls{VPL},
established on the class of \gls{VDSML}\cite{Salihbegovic2015}. The metamodel is
established on a formal representation and an abstract syntax. The runtime execution of
the generated files is running on top of the open-source project
OpenHAB\cite{eichstadt2015openhab}. The tool relies on a declarative \gls{DSL}
used to graphically represent an \gls{IoT} system.

\textbf{ArduinoML} is a modelling language use to describe the Arduino Uno
micro-controller\cite{Mosser2014}. The project has grown with the participation
of several students from the University of Nice–Sophia Antipolis in France.
ArduinoML is just a concept, his implementation is written with several
different languages such Haskell, Python, Pharo, Scala and many
others\cite{Mooser2017}.

\section{Summary}
\label{sec:iot_dev_summary}

In this chapter we have presented some of the challenges of the \gls{IoT} world.
Heterogeneity, power optimisation and connectivity are huge problems. Then we
presented some general programming languages available for the \gls{IoT}
development as well as some frameworks that simplify the design and
implementation of embedded devices. Finally, we have seen various \gls{DSL} who
tried to resolve challenges and problems encountered by the \gls{IoT}
projects.

The various \gls{DSL} we presented are solving some of the challenges of the
\gls{IoT}. It's not easy to solve all the challenges of the \gls{IoT} world, but
those \gls{DSL} offers a simple way to deal with some of them.

%%% Local Variables:
%%% mode: latex
%%% TeX-master: "../thesis"
%%% End: